\documentclass[fyp,12pt]{socreport}
\usepackage{fullpage}
\begin{document}
\pagenumbering{roman}
\title{Design and Implementation of an Algorithm for a Problem}
\author{Tan Ah Kow}
\projyear{2004/05}
\projnumber{H123456}
\advisor{Dr. Lee Ah Hua}
\deliverables{
	\item Report: 1 Volume
	\item Source Code: 1 DVD}
\maketitle
\begin{abstract}
The use of Wireless Sensor Networks for environmental monitoring has become
increasingly popular over the past decade due to its affordability, ease of deployment
and customisation, as well as its potentiality in the processing of sensed data. One of the
greatest challenges in this field would be in the design and implementation of an
efficient routing protocol which takes into account the various limitations of Wireless
Sensor Networks, such as battery life, limited storage capacities and high probability of
packet losses. Besides this, it is also extremely difficult to evaluate the performance of
such a protocol under crisis scenarios, due to its infrequency and unpredictability. In our
work, we have designed a routing protocol based on optimised Virtual Polar Coordinate
Routing (VPCR) (Newsome and Song, 2003) for use with our three-dimensional
testbed, comprising of 48 MICAz (Crossbow) motes spread across two floors of a
building. We have also developed a Java-based application with features for Event
Emulation and simple nodal analysis to assist us in our experiments. The overall
performance of our protocol will be gauged based on the average Path Stretch Factor
and path length comparisons between optimised and naïve VPCR.

\begin{descriptors}
 \item C5 - Computer System Implementation
 \item G2.2 - Graph Algorithms
\end{descriptors}
\begin{keywords}
	Problem, algorithm, implementation
\end{keywords}
\begin{implement}
	Solaris 10, g++ 3.3, Tcl/Tk 8.4.7
\end{implement}
\end{abstract}

\begin{acknowledgement}
   I would like to thank my friends, families and advisors.
   Without them, I would not have be able to complete this project.
\end{acknowledgement}

\listoffigures 
\listoftables
\tableofcontents 

\chapter{Introduction}
Many problems exist in computer science.  In this project, we 
studied one particular important problem and propose a solution 
for it.  

\section{Background}
In this section, we briefly discuss the history and background
of the problem.  A detail literature survey is presented in 
Chapter \ref{ch:related}.

The problem we study in this report is an important one.
This problem is first proposed in 1990 in the context
of graph theory \cite{smith90graph}.  Zhang gives the
first algorithm to the problem and applied it to solve several 
problems in artificial intelligence \cite{zhang91ai,zhang92ai}.  
More recently, a slightly different formulation of the problem
is studied independently \cite{kovsky92diff,ali94diff}.  None of this previous work
uses the technique that we propose in this project.  Thus, we 
believe that our algorithm is novel.

\section{The Problem}
In this section, we formally defined the problem.  We adopt
the definition given by Kovsky \cite{kovsky92diff}.

\section{Our Solution}
\section{Report Organization}

\chapter{Related Work}
\label{ch:related}

\chapter{Problem and Algorithm}
\section{Formal Description of Problem}
\section{Design of Algorithm}
\section{Proof of Correctness}
\section{Complexity Analysis}

\chapter{Evaluation}
\section{Implementation Details}
\section{Experimental Setup}
\section{Results}

\chapter{Conclusion}
\section{Contributions}
\section{Future Work}

\bibliographystyle{socreport}
\bibliography{report}

\appendix
\chapter{Code}

\chapter{Proofs}
In this appendix, we present alternate, longer, but more interesting proof 
of correctness of our algorithm.  This proof is based on induction and proof
by contradiction.
\end{document}
